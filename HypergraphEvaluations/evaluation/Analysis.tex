
\documentclass{article}
\usepackage{mathtools}
\usepackage{graphicx}
\usepackage{url}

\title{Qualitative Analysis: Hypergraph packages}
\author{Roscoe Casita}
\date{}
\begin{document}
\maketitle

\tableofcontents

\newpage
\section{Bioconductor: Hypergraph}

Location:\\
 \url{http://bioconductor.org/packages/release/bioc/html/hypergraph.html}\\

\begin{tabular}{| l |r | r |}
	\hline
	Restriction Description & Pro & Con \\
	\hline
	Programing Language: R & Configured for R Packages. & No other languages. \\
	Undirected \& Directed & Easy implementation &Non-repeating edges. \\
	List of edges & Easy implementation & $O(N)$ random time.\\
	Direct conversion to/from graph & Sparse matrix list & No generated support \\
	Static/Non-generated & Fast access &Must fit in main memory\\
	\hline
	\hline
	Odometer Support: None. & N/A & N/A \\
	Infinite/Generated Hypergraph: None. & N/A & N/A \\
	Iterative Functions: None. & N/A & N/A \\
	Rounded Space Mapping: None. & N/A & N/A \\
	\hline
\end{tabular}\\


\indent
The Bioconductor Hypergraph software package provides a Hypergraph template package that allows for simple standard graphs to be converted to hypergraph structures. The implementation uses a fixed list of hyperedge to construct the hypergraph. The size of the hypergraph is restricted to sizes that can fit in main memory. There is no support for generated / enumerated / projected hypergraphs. \\

This software package introduces a lightweight template that needs to be customized to the specific application. There are no standard algorithms  provided or enumeration techniques built into the interface.   

\newpage
\section{PaToHs: Hypergraph Partitioner}
\begin{tabular}{| l | r | r |}
		\hline
		Restriction Description & Pro & Con \\
		\hline
		Programming Language: C/C++ \& Matlab & Lightweight C & Tiny Library \\
		Graphs that need Cutting. & Highly optimized partitioner & One problem only.\\
		Array of Nodes \& Edges & Pure Functional & No support functions\\
		Hypergraph File Format & Read / Write routines provided & Arbitrary file format\\
		Integer Weights & Specilized partitioning & No generic algorithms.\\
		\hline
		\hline
		Odometer Support: ``Pins'' & User Managed & Parameters to functions\\
		Infinite/Generate Hypergraph: None & N/A & N/A \\
		Iterative Functions: Callbacks & Only Cut function & N/A\\
		Rounded Space Mapping: None & N/A & N/A \\
		\hline
\end{tabular}

\newpage
\section{Zoltan: Hypergraph Partitioner}
\begin{tabular}{| l | r | r |}
	\hline
	Restriction Description & Pro & Con \\
	\hline
	Programming Language: C++11 & Support Team & Larger code base \\
	Graphs that need Cutting. & Highly optimized partitioner & One problem only.\\
	Vector of Nodes \& Edges & Tilinos Integration & Non-std-template based.\\
	Specialized Graph Adapter & Easy partitioning mapping & No generic hypergraph.\\
	Weight for edge cut & Specilized for partitioning & No generic algorithms.\\
	\hline
	\hline
	Odometer Support: Primitive & Internal Maps Observed & No external access.\\
	Infinite/Generate Hypergraph: None & N/A & N/A \\
	Iterative Functions: Callbacks & Greedy Algorithms for NP & Only Weighted solutions.\\
	
	Rounded Space Mapping: None & N/A & N/A \\
	\hline
\end{tabular}
\end{document}